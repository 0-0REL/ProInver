\documentclass{report}
\usepackage[hidelinks]{hyperref}
\usepackage[spanish]{babel}
\usepackage[left=1.6cm, right=1.6cm, top=1.5cm, bottom=2cm]{geometry} % Margenes personalizados

\author{Rodrigo}
\title{ROS Jazzy}
\date{2025}

\begin{document}
\maketitle
\tableofcontents
\chapter{Lo básico y comandos}
Para listar cosas usar "list"
\begin{itemize}
    \item \texttt{ros2 node list} - Lista los nodos activos en el sistema.
    \item \texttt{ros2 topic list} - Muestra los tópicos activos que están siendo publicados o suscritos.
    \item \texttt{ros2 service list} - Lista los servicios disponibles en el sistema.
    \item \texttt{ros2 action list} - Muestra las acciones disponibles en el sistema.
\end{itemize}

\section{rqt}
Es una GUI para manipular ROS de forma mas visual, todo lo que puede hacer se puede hacer desde la 
terminal.

Se inicia poniendo \texttt{rqt} en la terminal con  para mostrar algo en rqt hay que poner
Plugins > Services > Service Caller desde la barra de menu.

\section{Nodos}
Los nodos se encargan de una tarea especifica, para ver que nodos hay activos se usa el comando:
\\\texttt{ros2 node list}
\\Para obtener todo el desglose del nodo se usa:
\\\texttt{ros2 node info <nombre del nodo>}

Esto regresa los siguientes tipos de información:
\begin{itemize}
    \item subscribers
    \item publishers
    \item services
    \item actions
\end{itemize}

\section{Tópicos}
Los tópicos son los canales de comunicación entre nodos, los nodos pueden publicar mediante los tópicos
o pueden suscribirse a ellos para recibir mensajes de otros nodos. Un nodo puede publicar y suscribirse
al mismo tiempo, la comunicación entre nodos no solo es uno a uno, sino también uno a muchos o muchos
muchos.
\\Para ver que tópicos hay activos se usa el comando:
\\\texttt{ros2 topic list} - Muestra los tópicos activos.
\\Se puede agregar el parámetro \texttt{-t} para mostrar el tipo del tópico.
\\\texttt{ros2 topic list -t} - Muestra los tópicos activos y su tipo.
\\Para ver a quien publica y quienes están suscritos se usa:
\\\texttt{ros2 topic info <nombre del tópico>} - Muestra la información del tópico.

También se pueden visualizar estas interacciones usando rqt, ya sea con \texttt{rqt} o \texttt{rqt\_graph},
desde rqt se puede llegar a graph con Plugins $>$ Introspection $>$ Node Graph.

Usando el comando:
\\\texttt{ros2 topic echo <nombre del tópico>}
\\Se puede ver el contenido del tópico mientras
se publica, esto es útil para ver que mensajes se están enviando y recibiendo en tiempo real.
\\Mediante el comando:
\\\texttt{ros2 interface show <nombre del tipo del tópico>}
\\Se puede ver la estructura del mensaje que se está enviando por el tópico.

Para publicar un mensaje se utiliza:
\\\texttt{ros2 topic pub <nombre del tópico> <tipo del mensaje> <mensaje>}
\\El mensaje se debe escribir en formato YAML, no es necesario escribir siempre todo el mensaje, solo de 
las partes que se van a modificar. Al enviar un mensaje vacío (sin la parte de mensaje) se envía el
mensaje por defecto.
\\Se puede usar el \textit{TAB} para autocompletar y ver opciones disponibles. Por defecto el mensaje
se envía en bucle infinito pero si se quiere enviar solo una vez se puede agregar el parámetro \texttt{--once}
\\\texttt{ros2 topic pub --once --w n <nombre del tópico> <tipo del mensaje> <mensaje>}
\\El parámetro \texttt{--w} n, es opcional pero indica que se esperan n numero de suscriptores antes
de enviar el mensaje.

La frecuencia de las publicaciones puede ver con el comando:
\\\texttt{ros2 topic hz <nombre del tópico>}
\\También se puede establecer la frecuencia agregando el parámetro \texttt{--rate <frecuencia>},
la frecuencia es en Hz.
\\\texttt{ros2 topic pub --rate <frecuencia> <nombre del tópico> <tipo del mensaje> <mensaje>}

Se puede monitorear el ancho de banda ocupado por un tópico con:
\\\texttt{ros2 topic bw <nombre del tópico>}

Además de pueden rastrear los tópicos que son de la tal tipo con:
\\\texttt{ros2 topic find <tipo del tópico>}

\appendix
\chapter{Abrir ros jazzy}
Para sourcear o lo equivalente a iniciarlo o abrirlo, se usa:
\\\texttt{source /opt/ros/jazzy/setup.bash}

\chapter{Problemas con el qsocket}
El wayland tiene problemas con el qsocket, hay que cambiar a x11 para resolver estos problemas.
\\Activar x11: \texttt{export QT\_QPA\_PLATFORM=xcb}
\\Para desactivar si es necesario: \texttt{unset QT\_QPA\_PLATFORM}
\end{document}