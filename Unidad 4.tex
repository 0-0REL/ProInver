\chapter{Estudio Económico-Financiero}
\section{Inversiones}
El monto de la inversión es la suma que el inversionista arriesga si acepta participar en un
proyecto, con los desembolsos; y que no se haran si no se lo ejecuta.
\subsection{Inicial}
La inversión inicial abarca la adquisición de todos los activos fijos o tangibles o deferidos
e intangibles necesarios para iniciar las operaciones de la empresa, con excepción del capital
de trabajo
\subsection{Fija}
Se entiende por activo tagible (que se puede tocar) o fijo, los bienes propiedades de la empresa,
tales como terrenos, edificios, maquinaria, equipo, mobiliario, vehiculos de trasporte, herramientas
entre otros. Se le llama fijo por que la empresa no puede desprenderse facilmente de él sin
que con ello ocasione problemas a sus actividades productivas (a diferencia del activo circulante).
\subsection{Diferida}
Se entiende por activo intagible el conjunto de bienes propiedad de la empresa necesarios para
el funcionamiento, entre ellos patentes de invención, marcas, diseños comerciales o industriales,
nombres comerciales, asistencia técnica o transferencia de tecnologia, gastos preoperativos
y de instalación y puesta en marcha, contratos de servicio (luz, teléfono, télex, agua, corriente
trifásica y servicios notariales), estudios que tienden a mejorar en el presente o en el futuro
el funcionamiento de la empresa, como estudios administrativos o de ingenieria, estudios de
evaluación, capacitación de personal dentro y fuera de la empresa, entre otros. En el caso
del costo del terreno, éste debe incluir: el precio de compra del lote, las comisiones a 
agentes, honorarios y gastos notariales, y aún el costo de demolición de estructuras existentes
que no se necesiten para los fines que se pretenda dar al terreno. En el caso del costo de
equipo y la maquinaria, debe verificarse si éste incluye fletes, instalaciones y puesta en
marcha. En la evaluaciónde proyectos se acostumbra presentar la lista de datos los activos
tangibles e intangibles, anotando que se incluya en cada uno de ellos.
\section{Determinación de costos de producción}
El cáluclo de los gastos o costos de producción se realiza asignado precios a los distintos
recursos requeridos, físicamente cuantificados de acuerdo con los estudios de ingeniría.

Sólo se considerará aquí la valoración a precios de mercado, señalando en los casos
pertinentes las informaciones que prodrían ser útiles y necesarias para la valoración social.

Para calcular y presentar los costos de producción de un proyecto se comienza por desglozarlos
en rubros parciales, de manera parecida a la empleada en la contabilidad de las empresas
que estan funcionando, para lo cual consideraremos.
\begin{itemize}
    \item Las materias primas que han de trasnformarse en satisfactores o bienes materiales.
    \item El trabajo necesario para trasnformar los materiales en productos. La suma de los
    costos de estos elementos forman el costo primo.
    \item A los factores anteriores hay que agregarles los equipos y maquinaria necesaria
    para la transformación de los materiales en productos, cuyas depreciaciones deben
    considerarse como un elemento del costo de la producción, al igual que el pago de la
    renta o depreciación de un edificio; los gastos generales de la fabricación tales como
    la energía, el alumbrado, el material indirecto, el trabajo directo, y otros que corresponden
    a la actividad productora en su conjunto, dificil de cargarse directamente a los productos.
    Todas estas partidas corresponden a los gastos indirectos de producción, construyendo el
    tercer elemento de los costos producción.
\end{itemize}

Estos conceptos pueden resumirse de la siguiente forma:
\begin{equation}
    \begin{array}{c} 
        \makecell{\text{Costo}\\\text{de}\\\text{producción}}
    \end{array}
    = 
    \begin{array}{|p{2.8cm}|p{2cm}|} % Define el ancho de las columnas
        \multicolumn{2}{c}{\text{Costo primo}} \\ \hline
        \makecell{Costo de la\\materia prima} & \makecell{Costo del\\trabajo}
    \end{array}
    +
    \begin{array}{c}
        \makecell{\text{Gastos}\\\text{de}\\\text{producción}}
    \end{array}
\end{equation}

Para lograr coordinar los diversos factores de la producción se necesita capacidad organizadora y
administrativa, así como un adecuado sistema de ventas, las cuales son auxiliares para obtener una
eficiente distribución de los productos; de ahí que se les considere costosos de distribución, cuya
formula se puede expresar de la siguiete forma.
\begin{equation}
    \begin{array}{c}
        \makecell{\text{Costos de}\\\text{distribución}}
    \end{array}
    =
    \begin{array}{c}
        \makecell{\text{Gastos de}\\\text{administración}}
    \end{array}
    +
    \begin{array}{c}
        \makecell{\text{Costo de}\\\text{venta}}
    \end{array}
\end{equation}

Por lo que el total estará integrado por:
\begin{equation}
    \begin{array}{c}
        \text{Costo total}
    \end{array}
    =
    \begin{array}{c}
        \makecell{\text{Costo de}\\\text{producción}}
    \end{array}
    +
    \begin{array}{c}
        \makecell{\text{Costo de}\\\text{distribución}}
    \end{array}
\end{equation}

Para obtener los precios de venta, es decir, el ingreso que debe cubrir el costo total más el margen de
utilidad, se le agrega un determinado persentaje de ganancia.
\begin{equation}
    \begin{array}{c}
        \makecell{\text{Precio de}\\\text{venta}}
    \end{array}
    =
    \begin{array}{c}
        \makecell{\text{Costo}\\\text{total}}
    \end{array}
    +
    \begin{array}{c}
        \makecell{\text{Porcentaje de}\\\text{utilidad}}
    \end{array}
\end{equation}

Los rubros que integran los costos de un proyecto pueden agruparse de la siguiente manera:
\begin{enumerate}
    \item Materias primas y otros materiales.
    \item Energía y combustible.
    \item Mano de obra.
    \item Seguros, impuestos y arriendos.
    \item Los gastos de ventas.
    \item Imprevistos y varios.
    \item Depreciación y obsolescencia.
\end{enumerate}

Supongamos ahora los siguientes costos e ingresos:

\vspace{10pt}
\begin{tabular}{p{5.8cm} c p{1.4cm}}
    \textbf{Costo} && \textbf{Total} \\
    Mano de obra directa & \textdollar & 20,000 \\
    Materiales directos && 25,000 \\
    \cline{3-3}
    Costo primo & \textdollar & 45,000 \\
    Costos indirectos de fabricación && 8,000 \\
    \cline{3-3}
    Costos de fabricación (total) & \textdollar & 53,000 \\
    Gastos administrativos && 14,200 \\
    \cline{3-3}
    Costos de producción & \textdollar & 67,000 \\
    Costos de ventas && 37,150 \\
    \cline{3-3}
    Costos totales & \textdollar & 104,350 \\
    Ingresos && 129,000 \\
    \cline{3-3}
    utilidad & \textdollar & 24,650
\end{tabular}
\section{Depreciación y amortización}
Con el transcurso del tiempo los activos tagibles renovables (por ejemplo, máquinas o edificios) sufren
una perdidad de valor que puede deberse a razones fisicas o económicas. La perdida del valor original
por causa del deterioro fisico o el desgaste a causa del uso, es lo que integra la depreciación
la cual solo se aplica al activo fijo.

La disminución de su valor surgida por causas económico se le conoce como absolencia.

La amortización por el contrario sólo se aplica a los activos diferidos o intangibles, ya que si hemos
adquirido una marca comercial, esta con el trasncurso del timpo no sufre una baja de precio o se deprecia,
por lo que el término amortización presenta el cargo anual que se debe efectuar para poder recuperar esa inversión.
Para realizar la sustitución de los equipos, se sugieren los siguientes criterios.

\textbf{El trabajo.} La vida de trabajo de una máquina depende de su productividad relativa. Se deprecia
en 20 años al 5\% anual.

\textbf{El desgaste.} Este depende de las condiciones de operación, su término es de 10 años al 10\% anual.

\textbf{El envejecimiento.} Este originado por la aparición de máquinas nuevas y mejores, para lo cual se
considera 7 años al 14\% anual.

Para calcular la depreciación los métodos más usados son: