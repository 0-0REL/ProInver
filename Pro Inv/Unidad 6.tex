\chapter{Evalucaión economica y analisis de sensibilidad}
\section{Rentabilidad}
De los diversos enigmas de aternativas multiples, los más determinantes y complicados en su
resolución son aquellos que aluden a inversiones propuestas en nuevos bienes, los cuales son
considerados como problemas de presupuestos de inversión. Estas dificultades son importantes,
y se consideran así no unicamente por el motivo de las grandes cantidades de capital por invertir,
sino por que la resolución que se efectue puede influir sobre la dirección de la empresa
durante años, y son complicados ya que requieren la estimación de las condiciones dominantes
durante varios años en el comportamiento futuro y también por que se requiere determinar la
asignación para las diferencias de oportunidad en que se producen los ingresos y egresos más
sobresalientes en lo que se refiere al capital.
\section{Valor presente neto}
La complicaión a solucionar en un asunto de inversión consiste en resolver si la inversión
en proposición se justifica por los ingresos que producira en el tiempo de su existencia. Un
emprendimiento invertira determinado capital en este momento tan sólo el espera para obtener
algo más de esa cantidad invertida en el futuro.

Las técnicas que se descrbierán son sistemas para comparar los siguientes componentes diferentes.
\begin{itemize}
    \item Una sola inversión, de manera de que una suma global, la cual se enfoca en el momento actual.
    \item Una sucesión de ingresos futuros.
\end{itemize}

El valor presente neto (VPN) es un indicador que se basa en el descuento de flujo de fondos, es decir,
el beneficio neto de cada período futuro descontada a una tasa dada para llegar a obtener el valor
presente. La suma de todos los ejercicios futuros neutralizados es el VPN. Los valores obtenidos en este
material se usaran en la hoja 75.
\[VPN=-X+\sum_{n = 1}^{n}\frac{A_n}{(1+i)^n}\]

A continuación podemos observar el siguiente flujo de fondos y el calculo de su correspondiente VPN.
\begin{example} Tasa de descuento $i=10\%=0.1$ 

    \centering{
        \begin{tabular}{c c c c c c}
            0 & 1 & 2 & 3 & 4 & 5 \\
            -130 & 10 & 40 & 50 & 60 & 70
        \end{tabular}}
    \[VPN=-130+\frac{10}{1.1}+\frac{40}{1.21}+\frac{50}{1.331}+\frac{60}{1.4641}+\frac{70}{1.61051}\approx34.16\]
\end{example}
\begin{example} Tasa de descuento $i=20\%=0.2$
        
     \centering{\begin{tabular}{c c c c c c}
            0 & 1 & 2 & 3 & 4 & 5 \\
            -450 & -10 & 110 & 150 & 190 & 230
        \end{tabular}}
    \[VPN=-450+\frac{-10}{1.2}+\frac{110}{1.44}+\frac{150}{1.728}+\frac{190}{2.0736}+\frac{230}{2.48832}\approx-110.07\]
\end{example}
\section{Tasa interna de rendimiento}